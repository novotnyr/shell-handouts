\documentclass[8pt,landscape]{extarticle}
\usepackage{fontspec}
\defaultfontfeatures{Ligatures=TeX}


\usepackage[landscape=true,paperwidth=9cm,paperheight=5cm,layoutwidth=9cm,layoutheight=5cm,layoutoffset={0cm,0cm},ignoreheadfoot=true,lmargin=3mm,rmargin=3mm,tmargin=3mm,bmargin=0mm]{geometry}
\usepackage{graphicx}
\usepackage{fancybox}
\usepackage{fancyvrb}
\usepackage{upquote}

\usepackage{tcolorbox}%
  \tcbuselibrary{xparse}
  
\usepackage{soul}
\sethlcolor{lightgray}

\pagestyle{empty}

\setromanfont{CamingoCode}
\setsansfont{CamingoCode}

\defaultfontfeatures{} % reset for mono font
\setmonofont[Ligatures=NoCommon]{CamingoCode}

\newfontfamily\Courier{Courier}%
\setlength{\unitlength}{1cm}
\setlength\parindent{0em}

\newcommand{\heading}[1]{%
\put(0,4.5){\makebox[9cm]{\textbf{\Large #1}}}%
}%

\newcommand{\Heading}[1]{%
{\begin{center}\bfseries\Large#1\end{center}}%
}%

\newcommand{\Em}[1]{\Ovalbox{\ttfamily #1}}
\newcommand{\EM}[1]{\fbox{#1}}

\def\VISIBLESPACE{{\tiny $\sqcup$}}

\newenvironment{karticka}[1]%
{%
\Heading{#1}%
}%
{%
\clearpage
}%

\DefineVerbatimEnvironment%
{obsah}{Verbatim}
{commandchars=\\\{\}}

\newtcolorbox{bottomcolorbox}{sharp corners, grow to left by=3mm, leftrule=0mm, rightrule=0mm, bottomrule=0mm, grow to right by=3mm, enlarge bottom by=-1mm}
\newtcolorbox{middlecolorbox}{sharp corners, grow to left by=3mm, grow to right by=3mm,leftrule=0mm, rightrule=0mm,  enlarge bottom by=-1mm}
\begin{document}
%-------------------------
\begin{karticka}{echo}
\begin{obsah}
slová oddelené medzerou na štandardný výstup
\end{obsah}
\begin{middlecolorbox}
\begin{Verbatim}
echo 'Hello World'
\end{Verbatim}
\end{middlecolorbox}
\end{karticka}
%-------------------------
\begin{karticka}{cat}
\begin{obsah}
výpis obsahu súboru na štandardný výstup
\end{obsah}
\begin{middlecolorbox}
\begin{Verbatim}
cat /etc/passwd
\end{Verbatim}
\end{middlecolorbox}
\end{karticka}
%-------------------------
\begin{karticka}{wc}
\begin{obsah}
počet riadkov / slov / znakov / bajtov
\end{obsah}
\begin{middlecolorbox}
\begin{Verbatim}
wc -l /etc/passwd
\end{Verbatim}
\end{middlecolorbox}
\end{karticka}
%-------------------------
\begin{karticka}{head}
\begin{obsah}
prvých \emph{N} riadkov súboru
\end{obsah}
\begin{middlecolorbox}
\begin{Verbatim}
head -n 10 /etc/passwd
\end{Verbatim}
\end{middlecolorbox}
\end{karticka}
%-------------------------
\begin{karticka}{tail}
\begin{obsah}
posledných \emph{N} riadkov súboru
\end{obsah}
\begin{middlecolorbox}
\begin{Verbatim}
tail -n 10 /etc/passwd
\end{Verbatim}
\end{middlecolorbox}
\begin{obsah}
od \emph{N}-tého riadka do konca
\end{obsah}
\begin{middlecolorbox}
\begin{Verbatim}
tail -n +10 /etc/passwd
\end{Verbatim}
\end{middlecolorbox}

\end{karticka}
%-------------------------
\begin{karticka}{cut}
\begin{obsah}
vysekávanie políčok podľa jednoznakového oddeľovača
\end{obsah}
\begin{middlecolorbox}
\begin{Verbatim}
cut -d: -f1 /etc/passwd
\end{Verbatim}
\end{middlecolorbox}
\end{karticka}
%-------------------------
\begin{karticka}{grep}
\begin{obsah}
vyhľadávanie a filtrovanie riadkov podľa regulár. výrazu
\end{obsah}
\begin{middlecolorbox}
\begin{Verbatim}
grep '^john' /etc/passwd
\end{Verbatim}
\end{middlecolorbox}

\begin{obsah}
\Em{-v} riadky bez zhody     \Em{-E} rozšírený regex
\Em{-i} ignoruje VEĽKÉ/malé  
\end{obsah}
\end{karticka}
%-------------------------
\begin{karticka}{awk}
\begin{obsah}
vylepšený cut s podporou viacerých oddeľovačov a 
pokročilých funkcií
\end{obsah}
\begin{middlecolorbox}
\begin{Verbatim}
awk -F: '{ print $1 }' /etc/passwd 
\end{Verbatim}
\end{middlecolorbox}
\begin{obsah}
\Em{-F} znaky oddeľovačov    
\Em{$1} prvé políčko v riadku
\end{obsah}

\end{karticka}
%-------------------------
\begin{karticka}{nl}
\begin{obsah}
očíslovanie riadkov
\end{obsah}
\begin{middlecolorbox}
\begin{Verbatim}
nl hello.c
\end{Verbatim}
\end{middlecolorbox}
\end{karticka}
%-------------------------
\begin{karticka}{sed}
\begin{obsah}
nahrádzanie textu v riadkoch
\end{obsah}
\begin{middlecolorbox}
\begin{Verbatim}
sed 's/root/admin/g' users.txt
\end{Verbatim}
\end{middlecolorbox}
\begin{obsah}
\Em{g} nahrádzanie všetkých výskytov v riadku
\Em{-E} zapne rozšírené regexy (GNU)
\end{obsah}

\end{karticka}
%-------------------------
\begin{karticka}{tr}
\begin{obsah}
nahrádzanie jednotlivých znakov, mazanie znakov
\end{obsah}
\begin{middlecolorbox}
\begin{Verbatim}
tr '_' '-' < files.txt
\end{Verbatim}
\end{middlecolorbox}
\begin{obsah}
\Em{-d} odstráni uvedené znaky
\end{obsah}

\end{karticka}
%-------------------------
\begin{karticka}{sort}
\begin{obsah}
triedenie podľa položiek
\end{obsah}
\begin{middlecolorbox}
\begin{Verbatim}
sort -t: -k3n /etc/passwd
\end{Verbatim}
\end{middlecolorbox}
\begin{obsah}
\Em{-t} oddeľovač políčok
\Em{-k} triedené políčko  \Em{n} číselné triedenie
\end{obsah}

\end{karticka}
%-------------------------
\begin{karticka}{uniq}
\begin{obsah}
zjednotí duplicitné riadky v zotriedenom vstupe
\end{obsah}
\vfill
\begin{bottomcolorbox}
sort names.txt | uniq
\end{bottomcolorbox}
\end{karticka}
%-------------------------
\begin{karticka}{printf}
\begin{obsah}
- vylepšené echo: podpora špeci znakov
- formátovaný výpis textu
\end{obsah}
\vfill
\begin{bottomcolorbox}
\begin{Verbatim}
printf 'Pouzivatel %s byva v %s \n' john /home/john
\end{Verbatim}
\end{bottomcolorbox}
\end{karticka}
%-------------------------
%-------------------------
\begin{karticka}{Nový skript}
- uvedený riadkom shebang 
\begin{middlecolorbox}
\begin{Verbatim}
#!/bin/sh
\end{Verbatim}
\end{middlecolorbox}
\par\vspace{1em}
- s atribútom executable

\begin{middlecolorbox}
\begin{Verbatim}
chmod +x skript.sh
\end{Verbatim}
\end{middlecolorbox}
\end{karticka}
%-------------------------
\begin{karticka}{POSIX Shell}
\begin{Verbatim}
- syntax shellu má milión dialektov
- POSIX: špecifikácia so zjednotenými vlastnosťami
         posixový skript pobeží všade
         (Linux, MacOS, AIX)
- bash:  najrozšírenejší shell
- ksh, zsh, fish: ďalšie shelly
\end{Verbatim}         
\end{karticka}
%-------------------------
\begin{karticka}{Podmienky}
\begin{picture}(2,1)
\put(0,0){%
\begin{minipage}{5cm}
if\VISIBLESPACE\Em{exit kód príkazu je nula}\\
then\\
...\\
else\\
...\\
fi
\end{minipage}
}
\put(+2,-1){%
\begin{minipage}{5cm}
\begin{tcolorbox}[sharp corners, grow to left by=3mm, grow to right by=3mm, leftrule=0mm, rightrule=0mm, enlarge bottom by=-1mm]
if grep root /etc/passwd\\
then\\
...\\
fi
\end{tcolorbox}
\end{minipage}
}
\end{picture}
\end{karticka}
%-------------------------
\begin{karticka}{Podmienky}
\begin{picture}(2,1)
\put(0,0){%
\begin{minipage}{5cm}
if\VISIBLESPACE[\VISIBLESPACE \hl{podmienka príkazu test}\VISIBLESPACE] \\
then\\
...\\
else\\
...\\
fi
\end{minipage}
}
\put(+2,-1){%
\begin{minipage}{5cm}
\begin{tcolorbox}[sharp corners, grow to left by=3mm, grow to right by=3mm, leftrule=0mm, rightrule=0mm, enlarge bottom by=-1mm]
-f: je to súbor?\\
-d: je to adresár?\\
-n: neprázdna premenná\\
-z: prázdna premenná\\
\phantom{ }=: porovnanie reťazcov
\end{tcolorbox}
\end{minipage}
}
\end{picture}
\vfill
\begin{tcolorbox}[sharp corners, grow to left by=3mm, leftrule=0mm, rightrule=0mm, bottomrule=0mm, grow to right by=3mm, enlarge bottom by=-1mm]
if\VISIBLESPACE[\VISIBLESPACE-f /etc/passwd\VISIBLESPACE]\\
then 
\end{tcolorbox}

\end{karticka}
%-------------------------
\begin{karticka}{Premenné: čítanie}
\begin{middlecolorbox}
\begin{Verbatim}
echo "$HOME"
\end{Verbatim}
\end{middlecolorbox}
\begin{obsah}
- uvedená dolárom
- obalená úvodzovkami
\end{obsah}
\end{karticka}
%-------------------------
\begin{karticka}{Premenné: zápis}
\begin{middlecolorbox}
\begin{Verbatim}
MENO='Grace Hopper'
\end{Verbatim}
\end{middlecolorbox}
\begin{obsah}
- reťazce do apostrofov
- žiadne medzery okolo \Em{=}
\end{obsah}

\end{karticka}
%-------------------------
\begin{karticka}{Premenné: z výstupu príkazu}
\begin{middlecolorbox}
\texttt{USERS="\$(\hl{wc -l < /etc/passwd})"}
\end{middlecolorbox}
\begin{obsah}
- \hl{\$(...)} zachytí štandardný výstup príkazu
- uvedieme do úvodzoviek
\end{obsah}

\end{karticka}
%-------------------------
\begin{karticka}{Premenné a úvodzovky}
- \Em{'v apostrofoch'} bežný reťazec\\
- \Em{"v úvodzovkách"} reťazec, ale\\ 
\phantom{xxxxxxxxxxxxxxxxxxxx}\Em{\$}, \Em{`}, \Em{\textbackslash}  
                 majú vlastný význam\\
- \Em{"\$HOME"}\ \ \ \ \ \ \ \ \      čítanie z premenných\\
- \Em{"Domov: \$HOME"} \ interpolácia \\
- \Em{`wc -l`} \ \ \ \ \ \ \ \        ekvivalent \$(wc -l)
\end{karticka}
%-------------------------
\begin{karticka}{Dostupné premenné}
\begin{obsah}
- \Em{1}, \Em{2}, atď: vstupné parametre
- \Em{HOME}:        domovský priečinok
- \Em{LOGNAME}:     login používateľa
- \Em{PWD}:         aktuálny adresár
- \Em{PATH}:        adresáre, kde sa hľadajú spustiteľné
                 programy
\end{obsah}
\end{karticka}
%-------------------------
\begin{karticka}{Cyklus for}
\begin{middlecolorbox}
for\VISIBLESPACE{}X\VISIBLESPACE{}in\VISIBLESPACE{}\Em{slová oddelené bielym miestom}\\
do\\
\phantom{XX}echo "\$X"\\
done
\end{middlecolorbox}
\par\vskip6pt
Ak sa \Em{in slová oddelené bielym miestom} vynechajú, \\
iteruje sa cez argumenty
\end{karticka}
%-------------------------
\begin{karticka}{Spracovanie súborov}
\vspace*{-3mm}
\begin{middlecolorbox}
\begin{Verbatim}[commandchars=+()]
for SUBOR in ./*.tex
do
  if [ -e "$SUBOR" ]
  then
    +Em(spracuj súbor v premennej SUBOR)
  fi
done
\end{Verbatim}
\end{middlecolorbox}
\Em{./*} pre prípad súborov začínajúcich pomlčkou \\
\Em{-e} lebo žolíky bez zhody expandujú sami na seba
\end{karticka}

%-------------------------
\begin{karticka}{Expanzia cesty}
\begin{tabular}{ll}
\Em{.} 		& aktuálny adresár\\
\Em{..} 		& rodičovský adresár\\
\Em{\textasciitilde}		& domovský priečinok\\
\Em{?}		& žolík pre jeden znak\\
\Em{*}		& žolík pre viacero znakov\\%
\Em{[abcd]}	& množina znakov
\end{tabular}
\end{karticka}
%------------
\begin{karticka}{find -- vyhľadávanie v podadesároch}
\begin{middlecolorbox}
\begin{Verbatim}
find . -name '*.c'
\end{Verbatim}
\end{middlecolorbox}
\begin{tabular}{ll}
\Em{.} & odkiaľ začať\\
\Em{-name \hl{'*.c'}} & podmienka\\
&\  \Em{-name} hľadanie podľa mena\\
&\  \Em{'*.c'} názov je v apostrofoch! \\
&\phantom{XXXXXXXXX} Je to argument pre find, \\
&\phantom{XXXXXXXXX} nie expanzia cesty!
\end{tabular}
\end{karticka}
%------------
\begin{karticka}{Spracovanie súborov}
\textbf{wc} s viacerými argumentami
\begin{middlecolorbox}
find . -exec \Em{wc -l \{\} +}
\end{middlecolorbox}
%
\textbf{wc} 1x pre každý súbor (staré, pomalé)
\begin{middlecolorbox}
find . -exec \Em{wc -l \{\} \textbackslash; }
\end{middlecolorbox}
\end{karticka}
%------------
\begin{karticka}{Riadky zo vstupu: xargs}
\begin{middlecolorbox}
seq 5 | xargs -I \% \Em{touch 'file\%.txt'}
\end{middlecolorbox}
\begin{obsah}
- Pre každý riadok zo vstupu sa vykoná príkaz. 
- Znak \Em{%} sa postupne nahrádza riadkom zo vstupu
  a vykonáva sa príkaz
- Častý zástupný znak: \Em{\{\}} (à la find)
\end{obsah}
\end{karticka}
%------------
\begin{karticka}{xargs folklór}
Alternatíva pre \textbf{find/exec}:
\begin{middlecolorbox}
find . | xargs -I \% basename \%
\end{middlecolorbox}
Spracovanie slov zo vstupu:
\begin{middlecolorbox}
< mena.txt xargs printf '\%s@bigcompany.com'
\end{middlecolorbox}
\end{karticka}
%-------------------------
\begin{karticka}{Cyklus while}
\begin{picture}(2,1)
\put(0,0){%
\begin{minipage}{8cm}
while\VISIBLESPACE\hl{príkaz s nulovým exit kódom}\\
do\\
...\\
done
\end{minipage}
}
\put(+2,-1){%
\begin{minipage}{5cm}
\begin{tcolorbox}[sharp corners, grow to left by=3mm, grow to right by=3mm, leftrule=0mm, rightrule=0mm, enlarge bottom by=-1mm]
while sleep 3\\
do\\
\phantom{XX}echo 'Ping!'\\
done
\end{tcolorbox}
\end{minipage}
}
\end{picture}
\end{karticka}
%------------
\begin{karticka}{Tipy pre hromadné spracovanie}
\begin{tabular}{lp{5cm}}
\Em{for} & súbory z jedného adresára,\par postupnosť príkazov nad nimi\\
\Em{for} & slová / parametre,\par postupnosť príkazov nad nimi\\
\Em{find/exec}& súbory zo stromu, \par jeden príkaz\\
\Em{xargs}& riadok/slovo zo stdin, \par jeden príkaz nad ním
\end{tabular}
\end{karticka}
%------------------------------
\Heading{Načítanie riadkov: read}
\begin{middlecolorbox}
read -r LINE 
\end{middlecolorbox}  
%
\begin{Verbatim}[commandchars=\\\{\}]
- načíta do premennej LINE jeden riadok 
  zo stdin
- ak sa riadok nenačíta, vráti nenulový
  exit kód
- parameter \fbox{-r} je vždy povinný
\end{Verbatim}
\newpage
%-------------------------
\begin{karticka}{while/read}
\begin{Verbatim}[commandchars=\\\{\}]
- načítavanie riadkov súboru do premenných
- políčka oddelené medzerou alebo obsahom premennej IFS
- \Em{while} iteruje, ak \Em{read} vracia nulový exit kód
- konvencia: dáta nesmú ísť z rúry, ale súboru!
\end{Verbatim}
%
\begin{middlecolorbox}
\begin{Verbatim}[commandchars=\\\{\}]
while\VISIBLESPACE\Em{IFS=:}\VISIBLESPACE\Em{read -r MENO PRIEZVISKO}
do
   echo "$MENO, $PRIEZVISKO"
done < mena.txt
\end{Verbatim}
\end{middlecolorbox}
\end{karticka}
%-------------------------
\begin{karticka}{Funkcie}
\begin{middlecolorbox}
\begin{Verbatim}
to_upper() {
  echo "$1" | tr [:lower:] [:upper:]
}
\end{Verbatim}
\end{middlecolorbox}
\begin{obsah}
- \Em{$1}, \Em{$2}... stringové argumenty funkcie
- návratová hodnota: výhradne číselný exit kód 
  (cez \Em{return})
- môže komunikovať cez stdin/stdout/stderr
\end{obsah}
\end{karticka}

%-------------------------
\begin{karticka}{Volanie funkcií}
\begin{obsah}
- funkcia je skript v skripte
- voláme ju bez zátvoriek

to_upper 'hello'

- presmerovanie výstupu do premennej takisto ako 
  pri bežnom príkaze

MESSAGE="$(to_upper 'hello')

\end{obsah}
\end{karticka}

%-------------------------
\begin{karticka}{Expanzie}
\begin{Verbatim}[commandchars=§\[\]]
§EM[~]       vlnky: domovský priečinok
               §EM[~ alebo ~root]
§EM[$( )]    príkazu: zachytenie výstupu príkazu
               §EM[LOGIN=$(logname)]
§EM[$(( ))]  aritmetická: základná matematika
               §EM[I=$((I + 1))]
§EM[${ }]    premennej: čítanie
               §EM[echo "${PATH}"]
\end{Verbatim}
\end{karticka}
%-------------------------
\begin{karticka}{Expanzie prázdnych premenných}
Ak je premenná \Em{1} prázdna:
\begin{table}[h!]
\begin{tabular}{lp{4cm}}
\$\{1:-default\}
&   nahradí sa default hodnotou
\\
\$\{1:=default\}
&   priradí sa do nej default hodnota
\\
\texttt{\$\{1:?'Chyba premenna'\}}
&   skript skončí s chybou a hláškou
\\
\$\{\#1\}
&
   dĺžka reťazca v premennej
\end{tabular}
\end{table}
\end{karticka}
%-------------------------
\begin{karticka}{Práca s reťazcami}
\textbf{predpis} je slovo, ktoré môže obsahovať žolíky
\begin{table}[h!]
\begin{tabular}{lp{6cm}}
\verb|${1%predpis}|
&
Odsekne najkratšiu príponu z konca
\\[6pt]
\verb|${1#predpis}|
&
Odsekne najkratšiu predponu zo začiatku
\\[6pt]
\verb|${1%%predpis}|
&
Odsekne najdlhšiu príponu
\\[6pt]
\verb|${1##predpis}|
&
Odsekne najdlhšiu predponu
\end{tabular}
\end{table}
\end{karticka}
%-------------------------
\begin{karticka}{Skladanie príkazov}
\begin{Verbatim}[commandchars=+\[\]]
- exit kódy možno považovať za true/false a skladať cez
  +EM[&&] a +EM[||]
- využíva sa skrátené vyhodnocovanie
- +EM[&&]: príkaz spusti, len ak predošlý príkaz uspel
- +EM[||]: ak príkaz zlyhá, spusti nasledovný príkaz
\end{Verbatim}
\end{karticka}
%-------------------------
\begin{karticka}{Skladanie príkazov}
\begin{Verbatim}[commandchars=+\{\}]
- oznám zlyhanie
+EM{grep "^alice" /etc/passwd || echo "Ziadna Alice"}

- založ adresár, ak neexistuje
+EM{[ -d ./cache ] || mkdir ./cache}

- zmaž súbor, ak existuje
+EM{[ -f .lock ] &&  rm .lock}
\end{Verbatim}
\end{karticka}
%-------------------------
\begin{karticka}{Triky s \&\&}
\begin{middlecolorbox}
príkaz1 \&\& príkaz2
\end{middlecolorbox}
\centering príkaz2 sa vykoná, len ak príkaz1 uspeje

\begin{table}[h!]\centering
\begin{tabular}{ccccc}
Príkaz 1 & \&\& & Príkaz 2 & = & Výsledok\\\hline
OK   & \&\& & OK & = & OK\\
FAIL & \&\& & nevykoná sa & = & FAIL\\
OK   & \&\& & FAIL & = & FAIL
\end{tabular}
\end{table}
\end{karticka}
%-------------------------
\begin{karticka}{Triky s ||}
\begin{middlecolorbox}
príkaz1 || príkaz2
\end{middlecolorbox}
\centering príkaz2 sa vykoná, ak príkaz1 zlyhá

\begin{table}[h!]\centering
\begin{tabular}{ccccc}
Príkaz 1 & || & Príkaz 2 & = & Výsledok\\\hline
OK     & || & nevykoná sa & = & OK\\
FAIL   & || & OK & = & OK\\
FAIL   & || & FAIL & = & FAIL
\end{tabular}
\end{table}
\end{karticka}

%-------------------------
\begin{karticka}{Zoznamy príkazov}
\begin{table}[h!]
\begin{tabular}{p{3.3cm}|p{4.5cm}}
\begin{tcolorbox}[sharp corners,left=0mm,right=0mm,leftrule=0mm,rightrule=0mm,toprule=0mm,bottomrule=0mm,enlarge bottom by=-5mm]
\begin{Verbatim}
príkaz1;príkaz2
\end{Verbatim}
\end{tcolorbox}
&
2 príkazy na jednom riadku
\\
\begin{tcolorbox}[sharp corners,left=0mm,right=0mm,leftrule=0mm,rightrule=0mm,toprule=0mm,bottomrule=0mm,enlarge bottom by=-5mm]
\begin{Verbatim}
príkaz1\
  príkaz2
\end{Verbatim}
\end{tcolorbox}
& 2 príkazy v jednom
\\
\begin{tcolorbox}[sharp corners,left=0mm,right=0mm,leftrule=0mm,rightrule=0mm,toprule=0mm,bottomrule=0mm,enlarge bottom by=-5mm]
\begin{Verbatim}
{príkaz1; príkaz2; }
\end{Verbatim}
\end{tcolorbox}
&
Viac príkazov  
sa tvári ako  
jeden pri 
presmerovaní  
vstupov 
a výstupov
\end{tabular}
\end{table}


\end{karticka}
%-------------------------
\begin{karticka}{Subshell}
\begin{obsah}
- shell spustí samostatný shell
- zdedia sa deskriptory súborov
- skopírujú sa premenné
  - zmeny premenných sa neprejavia v rodičovskom shelli
  - zmeny premennej v rúre sa neprejavia u rodiča
\end{obsah}

Subshelly nastanú:\par\vspace{1em}

\begin{tabular}{ll}
(príkaz1; príkaz2) & 2 skripty v izolácii
\\
príkaz1 | príkaz2 & spustenie príkazov v rúre\\
X=\$(príkaz) & zachytenie príkazu do premennej
\end{tabular}
\end{karticka}

%================ AWK ===================

%-------------------------
\begin{karticka}{Awk}
Pre každý riadok spĺňajúci \textbf{predpis} sa vykoná \textbf{akcia}
\begin{middlecolorbox}
predpis \{ akcia \}
\end{middlecolorbox}
Spustenie:

\begin{minipage}{4.3cm}
\begin{middlecolorbox}
\begin{ttfamily}
awk -F\Em{':'} '\{ print \}'
\end{ttfamily}
\end{middlecolorbox}
\end{minipage}
%
\hfill
%
\begin{minipage}{4.5cm}
\begin{middlecolorbox}
\begin{ttfamily}
awk -F\Em{':'} -f skript.awk
\end{ttfamily}
\end{middlecolorbox}
\end{minipage}

\Em{-F}: oddeľovač políčok

\end{karticka}
%-------------------------
\begin{karticka}{Predpisy awk}
\def\SEP{\{..\}}
\renewcommand{\arraystretch}{1.6}
\begin{tabular}{l@{\SEP\hspace{12pt}}l}
/regex/  & riadok spĺňa regex \\
NR=3    & tretí riadok  \\
\$3 > 3  & tretia položka > 3 \\
\$1 \textasciitilde\ /OK/ & prvá položka spĺňa regex \\
BEGIN  &  pred prvým riadkom \\
END      & po poslednom riadku \\
NR>3,/OK/  & kombinácia
\end{tabular}
\end{karticka}
%-------------------------
\begin{karticka}{Akcie awk}
%
\renewcommand{\arraystretch}{1.6}
\begin{tabular}{lp{5cm}}
\verb|{ print }| & vytlačí celý záznam/riadok \\
\verb|{ print $1 }| & vytlačí prvú položku \\
\verb|{ print $3, $1 }| & 3. a 1. položka oddelené výstupným oddeľovačom (medzera) \\
\verb|{ print "*" $3}| & konkatenácia medzerou  
\end{tabular}
\end{karticka}
%-------------------------
\begin{karticka}{Zabudované premenné awk}
%
\renewcommand{\arraystretch}{1.6}
\begin{tabular}{lp{5cm}}
\verb|$0|     & celý riadok \\
\verb|$1, $2| atď & obsah položiek na aktuálnom riadku\\
NR     & poradové číslo riadka\\
IFS    & oddeľovač políčok (viď \Em{-F})\\
OFS    & oddeľovač políčok na výstupe   \\ 
NF     & počet položiek v riadku
\end{tabular}
\end{karticka}
%
%-------------------
\begin{karticka}{Premenné awk}
%
\renewcommand{\arraystretch}{1.6}
\begin{tabular}{lp{5cm}}
\verb|IFS=","|     & printf IFS \\
\verb|MESSAGE="Hello"| & print HELLO\\
COUNT=0 & print COUNT
\end{tabular}
%
\par\medskip
awk rozpoznáva reťazce v úvodzovkách,
čísla \\ a asociatívne polia
\end{karticka}
%-------------------
\begin{karticka}{Funkcie awk}
%
\def\CRLF{\\[3pt]}
%
\Em{gsub(čo, čím, kde)} \\\phantom{XXX}
  Nahradenie reťazca v celom riadku \CRLF
\Em{sprint("format", parametre…)} \\\phantom{XXX}
  formátovanie a priradenie \CRLF
\Em{split(reťazec, do\_poľa)} \\\phantom{XXX}
  rozsekne reťazec do cieľového poľa \CRLF
\Em{getline} \\\phantom{XXX}
  načíta ďalší riadok
\end{karticka}
%-------------------
\begin{karticka}{Programovanie awk}
%
\def\CRLF{\\[3pt]}
%
Cyklus:
\begin{middlecolorbox}
  for (i = 0; i < NF; i++) \{ print i \}
\end{middlecolorbox}

Podmienka:

\begin{middlecolorbox}\ttfamily
  if ( COUNT > 0 ) \{ print "OK" \}
\end{middlecolorbox}
\end{karticka}
%-------------------
\begin{karticka}{Programovanie awk}
%
\def\CRLF{\\[3pt]}
%
Funkcia
\begin{middlecolorbox}
\begin{Verbatim}
  function sucet(x,y) { return x + y }
  sucet(2+3)
\end{Verbatim}  
\end{middlecolorbox}

Polia:

\begin{middlecolorbox}
\begin{Verbatim}
  a["John"] = 1
  a[0] = 1
\end{Verbatim}  
\end{middlecolorbox}
\end{karticka}

%-------------------
\begin{karticka}{sed -- spúšťanie}
%
\def\CRLF{\\[3pt]}
%
\textbf{sed \Em{program} súbor} \\\phantom{XXX}
  program priamo v riadku\CRLF
\textbf{sed -e \Em{program} -e \Em{program} \emph{atď} súbor} \\\phantom{XXX}
  viacero programov \CRLF
\textbf{sed -f \Em{program v súbore} súbor} \\\phantom{XXX}
  externý skript s programom \CRLF
\textbf{sed -n} \\\phantom{XXX}
  zruš implicitný výpis riadkov
\end{karticka}
%-------------------
\begin{karticka}{[s]ubstitute - nahraď}
\begin{middlecolorbox}
  s/\Em{čo}/\Em{čím}/g
\end{middlecolorbox}
\Em{čo} BRE regex. Pozor na obmedzenú syntax!\\
\Em{čím} BRE regex\\
\Em{g} nahrádzanie všetkých výskytov\\

s/pes/dog/  - nahraď prvý výskyt\\
s/:/;/g     – hromadné nahradenie

\end{karticka}
%-------------------
\begin{karticka}{[s]ubstitute - nahraď}
\renewcommand{\arraystretch}{1.52}
\begin{tabular}{p{2.5cm}p{5.5cm}}
\textbf{s/[0-9]//g}  \par\emph{Odstráň čísla} 
	& \Em{čo} je BRE regex, \Em{čím} môžeme vynechať.
\\
\textbf{s/pes/+\&+/g} \par\emph{Obaľ pluskami}
  & \Em{\&} reprezentuje nájdený text.
\\              
\textbf{s\#-\#*\#/g}    & Oddeľovač je mriežka.
\\
\textbf{s\#.* \textbackslash(.*\textbackslash)\#\textbackslash 1\#g} 
\par\emph{Nechaj len 2. slovo}
&
            Skupiny uzatvárame do 
            escapovaných zátvoriek. 
            Odkaz na 1. skupinu \Em{\textbackslash 1}
\end{tabular}
\end{karticka}
%-------------------
\begin{karticka}{Adresy}
\begin{middlecolorbox}
adresa1,adresa2\Em{príkaz}
\end{middlecolorbox}
\begin{obsah}
adresa: 
- číslo riadku. Posledný riadok: \Em{\$}
- /regex/
Príkazy podľa typu berú 0, 1 alebo 2 adresy.
\end{obsah}
\end{karticka}
%-------------------
\begin{karticka}{[p]rint - tlač}
\renewcommand{\arraystretch}{1.52}
\begin{tabular}{p{2.5cm}p{5.5cm}}
p            & tlač každý riadok 2x 
               (raz implicitne, raz  
               explicitne)\\
sed -n 1p    & len 1. riadok. Implicitný  
               výpis vypnutý
               \\
\verb|sed -n 1,5p| & prvých 5 riadkov (=head)
\\
\verb|sed -n '3,$'|   & od 3. riadku do konca (pozor na \$)
\\
\verb|sed -n /#/p|  & len riadky s \# (=grep)
\end{tabular}
\end{karticka}
%-------------------
\begin{karticka}{[d]elete - maž}
\def\CRLF{\\[3pt]}
\def\NEWLINE{\\\phantom{XXX}}
\newcommand{\PRIKAZ}[1]{\textbf{#1}\\\phantom{XXX}}

\PRIKAZ{1,3d}
vymaže prvé 3 riadky\CRLF
\PRIKAZ{6,\$d}
vymaže od 6. riadku do konca \\
nechá prvých 5 riadkov\CRLF
\PRIKAZ{/\#/d}
vymaže riadky začínajúce \#

\end{karticka}
%-------------------
\begin{karticka}{[i]nsert, [a]ppend, [c]hange}
\def\CRLF{\\[3pt]}
\def\NEWLINE{\\\phantom{XXX}}
\newcommand{\PRIKAZ}[1]{\textbf{#1}\\\phantom{XXX}}

\PRIKAZ{/public class/\Em{i} /* @author jp */}
vloží pred riadok daný text\CRLF
\PRIKAZ{/public class/\Em{a} /* class */}
vloží za riadok daný text\CRLF
\PRIKAZ{1,3\Em{c}---------------}
zamení prvé tri riadky za čiaru\CRLF
\PRIKAZ{/\textasciicircum\#/\Em{c}------}
zamení riadok začínajúci mriežkou za čiaru

\end{karticka}
%-------------------
\begin{karticka}{Viacriadkové skripty pre sed}
\begin{middlecolorbox}
sed -e \Em{'s/pes/dog/'} -e \Em{'s/vlk/wolf/'}
\end{middlecolorbox}
Riadok postupne putuje príkazmi.
\end{karticka}
%-------------------
\begin{karticka}{Externé skripty pre sed}
\begin{middlecolorbox}
\begin{Verbatim}
1,3 {
  s/pes/dog/
  s/vlk/wolf
}
\end{Verbatim}
\end{middlecolorbox}
\par\medskip
V súboroch. Zavádzame parametrom \Em{-f}
\vfill
\begin{bottomcolorbox}
\begin{Verbatim}
sed -f skript.sed
\end{Verbatim}
\end{bottomcolorbox}
\end{karticka}
%-------------------
\begin{karticka}{sed - zriedkavé príkazy}
\begin{obsah}
= čísluje riadky
n načíta ďalší riadok
N prilepí ďalší riadok k aktuálnemu
\end{obsah}
\end{karticka}


\end{document}